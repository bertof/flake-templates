\usetheme{/sesarbeamer}

\usepackage{cite}
\usepackage{soul}

\usepackage{qrcode}
\usepackage{fontawesome,academicons}
\usepackage{verbatim}

\usepackage{adjustbox}
\usepackage[acronym]{glossaries}

% \newacronym{API}{API}{Application Programming Interface}
% \newacronym{CA}{CA}{Certification Agent}
% \newacronym{CDN}{CDN}{Content Distribution Network}
% \newacronym{CLI}{CLI}{Command Line Interface}
% \newacronym{CSV}{CSV}{Comma Separated Values}
% \newacronym{DAG}{DAG}{Directed Acyclic Graph}
% \newacronym{DNS}{DNS}{Domain Name System}
% \newacronym{FIFO}{FIFO}{First In First Out}
% \newacronym{GLB}{GLB}{Greatest Lower Bound}
% \newacronym{HTTP}{HTTP}{HyperText Transfer Protocol}
% \newacronym{HTTPS}{HTTPS}{HyperText Transfer Protocol over Secure Socket Layer}
% \newacronym{ISP}{ISP}{Internet Service Provider}
% \newacronym{IXP}{IXP}{Internet Exchange Point}
% \newacronym{LRU}{LRU}{Least Recently Used}
% \newacronym{LUB}{LUB}{Least Upper Bound}
% \newacronym{PIT}{PIT}{Pending Interest Table}
\newacronym{QoS}{QoS}{Quality of Service}
% \newacronym{RTT}{RTT}{Round Trip Time}
% \newacronym{TCP}{TCP}{Transmission Control Protocol}
% \newacronym{TLS}{TLS}{Transport Layer Security}
% \newacronym{URI}{URI}{Uniform Resource Identifier}
\newacronym{KPI}{KPI}{Key Performance Indicator}

% 5G system components
\newacronym{AF}{AF}{Application Function}
\newacronym{AM}{AM}{Access and Mobility Function}
\newacronym{AMF}{AMF}{Access and mobility Management Function}
\newacronym{AUSF}{AUSF}{Authentication Server Function}
\newacronym{CP}{CP}{Control Plane}
\newacronym{CPF}{CPF}{Control Plane Function}
\newacronym{NEF}{NEF}{Network Exposure Function}
\newacronym{NF}{NF}{Network Function}
\newacronym{NRF}{NRF}{Network Repository Function}
\newacronym{NSSAI}{NSSAI}{Network Slice Selection Assistance Information}
\newacronym{NSSF}{NSSF}{Network Slice Selection Function}
\newacronym{NSS}{NSS}{Network Slice Selection Function}
\newacronym{PCF}{PCF}{Policy Control Function}
\newacronym{SCP}{SCP}{Service Communication Proxy}
\newacronym{SMF}{SMF}{Session Management Function}
\newacronym{UDM}{UDM}{Unified Data Management}
\newacronym{UDR}{UDR}{Unified Data Repository}
\newacronym{UE}{UE}{User Equipment}
\newacronym{UPF}{UPF}{User Plane Function}
\newacronym{UP}{UP}{User Plane}

\newacronym{CSMF}{CSMF}{Communication Service Management Function}
\newacronym{EAP}{EAP}{Extensible Authentication Protocol}
\newacronym{MEAO}{MEAO}{Mobile Edge Application Orchestrator}
\newacronym{MEC}{MEC}{Multi-access Edge Computing}
\newacronym{MEP}{MEP}{Mobile Edge Platform}
\newacronym{MEPM}{MEPM}{Mobile Edge Platform Manager}
\newacronym{NSSMF}{NSSMF}{Network Slice Subnet Management Function}
\newacronym{RCS}{RCS}{Rich Communication Services}
\newacronym{SLA}{SLA}{Service Level Agreement}
\newacronym{SLO}{SLO}{Service Level Objective}
\newacronym{VIM}{VIM}{Virtualization Infrastructure Manager}
\newacronym{VNFD}{VNFD}{Virtualized Network Function Descriptor}
\newacronym{VNFO}{VNFO}{Virtualized Network Function Orchestrator}
\newacronym{VNF}{VNF}{Virtualized Network Function}


\usepackage{shellesc}
\usepackage{tikz}
\usetikzlibrary{external}
\tikzexternalize % activate and define .cache/ as cache folder
\usetikzlibrary{
	calc,
	arrows.meta,
	backgrounds,
	calc,
	fit,
	positioning,
	shadows,
	shapes.arrows,
	shapes.geometric,
	shapes.symbols
}

\newcommand{\inputtikz}[1]{%
	\tikzsetnextfilename{#1}%
	\input{#1}%
}

\tikzset{%
	base/.style = {
			% minimum width=1cm, 
			% minimum height=1cm, 
			text centered,
			% font=\sffamily,
			inner sep=0.5em,
		},
	boxes/.style = {
			base,
			draw=black,
			rectangle,
			rounded corners
		},
	arrow_label/.style = {
			base,
			midway,
			% sloped,
			fill=white, anchor=center,
		},
	on slide/.code args={<#1>#2}{%
			\only<#1>{\pgfkeysalso{#2}}%
		},
}


% meta-data
\title{QoS-aware Deployment of Service Compositions in 5G-empowered Edge-Cloud Continuum}
% \subtitle{A critical overview on protocols for cloud-ready applications}
\author{
	\href{mailto:marco.anisetti@unimi.it}{Marco Anisetti},
	\href{mailto:filippo.berto@unimi.it}{\textbf{Filippo Berto}},
	\href{mailto:ruslan.bondaruc@unimi.it}{Ruslan Bondaruc}\\
	University of Milan, Italy\\
	\{firstname.lastname\}@unimi.it
}
\date{July 5, 2023}
% \titlebackground{images/background}

% document body
\begin{document}

\maketitle

\section{Introduction}

\section{Scenario, Requirements and Architecture}

\subsection{Requirements}

\subsection{Deployment Architecture}

\subsection{Cloud}

\subsection{Telco Edge}

\subsection{On-premises}

\section{Methodology}

\subsection{Annotated Service Composition Template}

\subsection{Annotated Continuum Facilities Graph}

\subsection{Deployment Matching}

\subsection{Deployment Recipes}

\section{Experimental Evaluation}

\subsection{Experimental Setup}

\subsection{Performance Evaluation}

\section{Conclusions}

\backmatter

% \section{A bit on myself}

% \begin{frame}{Filippo Berto}
% 	Ph.D. candidate in Computer Science @ SESAR Lab, University of Milan\vspace{1em}

% 	Working on:
% 	\begin{itemize}
% 		\item Formal methods for Security/QoS assurance of infrastructures and services
% 		\item Cloud-edge computing infrastructures
% 		\item 5G services and 5G-enabled edge computing
% 	\end{itemize}\vspace{1em}

% 	Having fun with distributed computing, cloud networks, programming languages for safety critical applications, embedded programming
% \end{frame}

% \qrslide{LINK TO THE SLIDES}{https://bit.ly/ct-protocols}

% \section{Application layer protocols}

% \begin{frame}
% 	\centering
% 	{\Large\faPencil\quad What is a communication protocol?}\vspace{1em}

% 	\uncover<2->{
% 		A communication protocol is a set of \alert{rules} and \alert{conventions} that govern the exchange of \alert{information} between two or more entities, ensuring \alert{consistent} and \alert{structured} communication.
% 	}
% \end{frame}

% \begin{frame}{Where are we}
% 	\begin{columns}
% 		\begin{column}{.6\textwidth}
% 			High level protocols:\vspace{1em}
% 			\begin{itemize}
% 				\item We rely on the lower level infrastructure (caching, SDNs, routing, \ldots)
% 				\item Simpler abstractions for the developer
% 				\item We can concentrate on the data and its usage in the services
% 			\end{itemize}
% 		\end{column}
% 		\begin{column}{.3\textwidth}
% 			\inputtikz{images/iso_osi.tex}
% 		\end{column}
% 	\end{columns}
% \end{frame}

% \begin{frame}{Today's question}
% 	I'm building a new service that communicates with other services (web UI, smartphone application, IoT device, worker nodes, \ldots).\vspace{1em}

% 	\begin{center}
% 		\Large
% 		\uncover<2->{\faPencil\quad Which protocols should I use?} \only<3->{And how?}
% 	\end{center}
% \end{frame}

% \section{HTTP-based}

% \begin{frame}{How good are they?}
% 	We can use HTTP as our carrier protocol: \vspace{1em}

% 	\begin{columns}[t]
% 		\begin{column}{.5\textwidth}
% 			{\large \alert{Advantages}}
% 			\begin{itemize}
% 				\item It's a very versatile protocol (handles binary and text-encoded data)
% 				\item HTTP is (mostly) a simple and versatile protocol
% 				\item Easily integrates with web browsers/applications
% 				\item Handles download and upload of binary files
% 				\item Newer versions also handle streams
% 			\end{itemize}
% 		\end{column}
% 		\begin{column}{.5\textwidth}
% 			{\large \alert{Disadvantages}}
% 			\begin{itemize}
% 				\item Specification is lager and more complex each version
% 				\item Security delegated to transport layer
% 				\item A connection is kept open until for the full session
% 				\item Text-encoded information has to be parsed
% 				\item Stateless protocol: state management is delegated to the application
% 				\item Caching may introduce inconsistent behavior
% 			\end{itemize}
% 		\end{column}
% 	\end{columns}
% \end{frame}

% \section{SOAP}

% \begin{frame}[fragile]{Request}
% 	Soap is an XML based protocol for exchanging structured data. It can be encapsulated with HTTP and other protocols. \vspace{1em}

% 	\alert{Request}:
% 	\scriptsize
% 	\begin{verbatim}
% 		POST /StockQuote HTTP/1.1
% 		Host: www.stockquoteserver.com
% 		Content-Type: text/xml;
% 		charset="utf-8"
% 		Content-Length: nnnn
% 		SOAPAction: "Some-URI"
% 		<?xml version="1.0"?>
% 		<SOAP-ENV:Envelope
% 		    xmlns:SOAP-ENV="http://schemas.xmlsoap.org/soap/envelope/"
% 		    SOAP-ENV:encodingStyle="http://schemas.xmlsoap.org/soap/encoding/">
% 		    <SOAP-ENV:Body>
% 		        <m:GetLastTradePrice xmlns:m="Some-URI">
% 		            <symbol>DIS</symbol>
% 		        </m:GetLastTradePrice>
% 		    </SOAP-ENV:Body>
% 		</SOAP-ENV:Envelope>
% 	\end{verbatim}
% \end{frame}

% \begin{frame}[fragile]{Response}
% 	Soap is an XML based protocol for exchanging structured data. It can be encapsulated with HTTP and other protocols. \vspace{1em}

% 	\alert{Response}:
% 	\scriptsize
% 	\begin{verbatim}
% 		HTTP/1.1 200 OK
% 		Content-Type: text/xml;
% 		charset="utf-8"
% 		Content-Length: nnnn
% 		<?xml version="1.0"?>
% 		<SOAP-ENV:Envelope
% 		  xmlns:SOAP-ENV="http://schemas.xmlsoap.org/soap/envelope/"
% 		  SOAP-ENV:encodingStyle="http://schemas.xmlsoap.org/soap/encoding/"/>
% 		  <SOAP-ENV:Body>
% 		    <m:GetLastTradePriceResponse xmlns:m="Some-URI">
% 		      <Price>34.5</Price>
% 		    </m:GetLastTradePriceResponse>
% 		  </SOAP-ENV:Body>
% 		</SOAP-ENV:Envelope>
% 	\end{verbatim}
% \end{frame}

% \begin{frame}[fragile]{Critique}
% 	SOAP has many disadvantages:
% 	\vspace{1em}
% 	\begin{itemize}
% 		\item HUGE overhead
% 		\item Complex implementation with strict semantics
% 		\item Hard to parse (and read)
% 		\item Error-prone if human-generated
% 	\end{itemize}
% 	\vspace{1em}
% 	Use it only for backward compatibility with older systems.
% \end{frame}

% \section{REST}

% \begin{frame}[fragile]{Request \& Response}
% 	\begin{columns}
% 		\begin{column}{.5\textwidth}
% 			\begin{itemize}
% 				\item \textbf{\texttt{GET}}: HTTP method (or verb), describes the intended action
% 				\item \textbf{\texttt{/api/users/123}}: endpoint of the request that indicates the resource
% 				\item \textbf{\texttt{example.com}}: hostname of the server, can be used to differentiate and redirect requests
% 				\item \textbf{\texttt{application/json}}: format of the response requested by the client and provided by the server
% 			\end{itemize}
% 		\end{column}
% 		\begin{column}{.5\textwidth}
% 			\alert{Request}:
% 			{
% 			\footnotesize
% 			\begin{verbatim}
% 				GET /api/users/123 HTTP/1.1
% 				Host: example.com
% 				Accept: application/json
% 			\end{verbatim}
% 			}
% 			\vspace{1em}

% 			\alert{Response}:
% 			{
% 			\footnotesize
% 			\begin{verbatim}
% 				HTTP/1.1 200 OK
% 				Content-Type: application/json

% 				{
% 				  "id": 123,
% 				  "name": "John Doe",
% 				  "email": "johndoe@example.com"
% 				}
% 			\end{verbatim}
% 			}
% 		\end{column}
% 	\end{columns}
% \end{frame}

% \begin{frame}{Verbs}
% 	REST uses HTTP verbs and URL components to define semantics on requests.
% 	\vspace{1em}
% 	\begin{itemize}
% 		\item \alert{\textbf{GET}}: used to retrieve a representation of a resource or a collection of resources.
% 		\item \alert{\textbf{POST}}: used to submit data to be processed and create a new resource.
% 		\item \alert{\textbf{PUT}}: used to update or replace an existing resource with the data sent in the request.
% 		\item \alert{\textbf{PATCH}}: used to partially update an existing resource.
% 		\item \alert{\textbf{DELETE}}: used to remove a specific resource identified by its URL or identifier.
% 	\end{itemize}
% 	\vspace{1em}
% 	GET should not have any side effects. GET and PUT should be idempotent.
% \end{frame}

% \begin{frame}{Endpoints}
% 	Endpoints in REST model resources as a tree, where each component is a branch.
% 	\uncover<2->{Resource model $\not=$ data model}.
% 	\vspace{1em}
% 	\begin{itemize}[<+(2)->]
% 		\item \alert{\texttt{/users}}: a collection of users (i.e., usernames)
% 		\item \alert{\texttt{/users/:username}}: a specific user
% 		\item \alert{\texttt{/users/:username/profile\_picture}}: a specific user's profile picture
% 		\item \alert{\texttt{/users/:username/posts}}: a specific user's collection of posts (i.e., IDs)
% 	\end{itemize}
% 	\vspace{1em}
% 	Having a consistent and robust naming strategy improves transparency and readability.
% \end{frame}

% \begin{frame}{Endpoints best practices}
% 	Some best practices for your resource models:
% 	\vspace{1em}
% 	\begin{itemize}[<+->]
% 		\item Collections of resources use plural names \codesample{GET /device-manager/devices}
% 		\item Use dashes (\texttt{-}) to improve readability \codesample{GET /device-manager/devices/:id}
% 		\item Don't use trailing slashes (\texttt{/}) \codesample{GET /device-manager/devices/:id\alert{/}}
% 		\item Use only lowercase letters \codesample{GET /ThIsIsUnReadable/pLeasE/DonT}
% 	\end{itemize}
% 	\vspace{1em}
% 	Having a consistent and robust naming strategy improves transparency and readability.
% \end{frame}

% \begin{frame}{Query arguments}
% 	Query arguments are \alert{optional} arguments that are used to filter the results:
% 	\vspace{1em}
% 	\begin{itemize}[<+->]
% 		\item \codesample{GET /device-manager/devices\alert{?active=true}}
% 		\item \codesample{GET /users\alert{?follow=user1,user2}}
% 		\item \codesample{GET /users/:username/posts\alert{?limit=25\&page=3}}
% 	\end{itemize}
% 	\vspace{1em}
% 	Query parameters should be optional. If not, use body arguments or change your resource model.
% \end{frame}

% \begin{frame}[fragile]{JSON as body}
% 	The most common data format used in REST requests is JSON (defined in ECMA-404), a simple and lightweight serialization format that can describe:
% 	\begin{columns}
% 		\begin{column}{.5\textwidth}
% 			\begin{itemize}
% 				\item Integers
% 				\item Floating point numbers
% 				\item Strings
% 				\item Lists of values
% 				\item String-value maps
% 			\end{itemize}
% 		\end{column}
% 		\begin{column}{.5\textwidth}
% 			\begin{verbatim}
% 				{
% 				    "id": 123,
% 				    "name": "John Doe",
% 				    "email": "johndoe@example.com",
% 				    "firends": [314,2425,242],
% 				    "popularity": 123.34
% 				}
% 			\end{verbatim}
% 		\end{column}
% 	\end{columns}
% 	\vspace{1em}
% 	JSON is a fairly efficient format that is easy to parse and to read.

% 	Use JSON-encoded request and response bodies to combine multiple information.
% \end{frame}

% \begin{frame}{Some best practices}
% 	Some best practices for your resource models:
% 	\vspace{1em}
% 	\begin{itemize}[<+->]
% 		\item Use nouns to represent resources and HTTP verbs to indicate actions
% 		\item ``Store'' resources use fixed URIs for their contents, don't change them after they're once set \codesample{GET /stickers/:id/preview}
% 		\item ``Controller'' resources expose interactions as resources using verbs \codesample{POST /voting-pool/:id/vote}
% 	\end{itemize}
% 	\vspace{1em}
% 	Having a consistent and robust naming strategy improves transparency and readability.
% \end{frame}

% \section{GraphQL}

% \begin{frame}{What happens when systems become complex?}
% 	\begin{itemize}[<+->]
% 		\item Data collection becomes \alert{complex}
% 		      \begin{itemize}[<.->]
% 			      \item Parallelize requests at the expense of \alert{latency} and \alert{bandwidth} usage
% 		      \end{itemize}
% 		\item Resources become complex
% 		      \begin{itemize}[<.->]
% 			      \item We split resources in larger hierarchies $\rightarrow$ more complex resource model
% 			      \item We only want part of the resources body $\rightarrow$ query hell
% 		      \end{itemize}
% 		\item Documentation and version management?
% 	\end{itemize}
% \end{frame}

% \begin{frame}{A strongly typed language for data querying}
% 	\begin{itemize}[<+->]
% 		\item GraphQL uses \alert{strong type}-based hierarchies to define the resource model.
% 		\item Resources have \alert{attributes} and \alert{methods} like in common object-oriented programming languages.
% 		\item Clients specify exactly \alert{what data} they need and the \alert{shape} of the response.
% 		\item GraphQL also supports data updates through \alert{``mutations''}.
% 		\item Clients can fetch \alert{real-time updates} through subscriptions, simplifying querying live data sources.
% 		\item GraphQL framework expose the \alert{resource schema}, allowing for consistent \alert{code-driven documentation}.
% 	\end{itemize}
% \end{frame}

% \begin{frame}[fragile]{What about performance?}
% 	\begin{columns}
% 		\begin{column}{.4\textwidth}
% 			GraphQL allows you to \alert{combine queries}, reducing the number of network requests and moving most of the execution to the \alert{server side}. \vspace{1em}

% 			Moreover, the hierarchical structure of resources allows \alert{parallel evaluation}, where possible.
% 		\end{column}
% 		\begin{column}{.6\textwidth}
% 			\scriptsize
% 			\begin{verbatim}
%                 {
%                   location(id: "32") {
%                     id
%                     name
%                     latitude
%                     longitude
%                     altitude
%                     measurements(paging: { limit: 10 }) {
%                       id
%                       value
%                       sensor {
%                         sensorType
%                         unit
%                       }
%                     }
%                   }
%                 }
% 			\end{verbatim}
% 		\end{column}
% 	\end{columns}
% \end{frame}

% \begin{frame}[fragile]{Mutations}
% 	\begin{columns}
% 		\begin{column}{.4\textwidth}
% 			Data update is achieved through mutations.

% 			Moreover, the hierarchical structure of resources allows \alert{parallel evaluation}, where possible.
% 		\end{column}
% 		\begin{column}{.6\textwidth}
% 			\scriptsize
% 			\begin{verbatim}
% 				mutation insertTestMeasurement {
% 				    insertMeasurements(measurements:[{
% 				        insertionAgent:"test-agent",
% 				        location:  {id: "test-location"},
% 				        sensor: {id: "test-sensor", sensorType: "test-sensor"},
% 				        value: 123456.789,
% 				    }])
% 				}
% 			\end{verbatim}
% 		\end{column}
% 	\end{columns}
% \end{frame}

% \qrslide{DEMO}{https://bit.ly/3BQKwXq}

% \section{gRPC}

% \begin{frame}{Protobuf based wire protocol}
% 	\begin{center}
% 		{\Large\faPencil\quad What if the focus is performance?}
% 	\end{center}
% 	\vspace{1em}

% 	We move to wire-based protocols:
% 	\begin{itemize}
% 		\item \alert{Binary representation} saves us time in the parsing step
% 		\item \alert{Out-of-message protocol definition} saves more space
% 		\item (Close to) optimal \alert{data representation} and \alert{packing}
% 	\end{itemize}

% 	gRPC is one of the most used in the industry for its performance and features.
% \end{frame}

% \begin{frame}{Key features}
% 	\begin{itemize}[<+->]
% 		\item \alert{Fast \& efficient}: uses Protobuf for interface definition and binary serialization format
% 		\item \alert{Bi-directional Streaming}: supports unary and streaming communication patterns
% 		\item \alert{Support for multiple transport protocols}: default is HTTP/2, with support for multiplexing, flow control and header compression
% 		\item \alert{Code and documentation generation}: language-specific protocol libraries are generated at compile time, ensuring type safe communication
% 		\item \alert{Default values are omitted}: values are encoded only if they differ from the default, saving space
% 	\end{itemize}
% \end{frame}

% \begin{frame}[fragile]{Example protocol definition}
% 	\begin{columns}
% 		\begin{column}{.7\textwidth}
% 			\scriptsize
% 			\begin{verbatim}
% 				syntax = "proto3";

% 				package example;

% 				service UserService {
% 				  rpc GetUser(UserRequest) returns (UserResponse) {}
% 				  rpc UpdateUser(UserUpdateRequest) returns (UserResponse) {}
% 				  rpc DeleteUser(UserDeleteRequest) returns (UserResponse) {}
% 				}

% 				message UserRequest {
% 				  string user_id = 1;
% 				}

% 				message UserUpdateRequest {
% 				  string user_id = 1;
% 				  string name = 2;
% 				  int32 age = 3;
% 				}
% 			\end{verbatim}
% 		\end{column}
% 		\begin{column}{.3\textwidth}
% 			\scriptsize
% 			\begin{verbatim}
% 				message UserDeleteRequest {
% 				  string user_id = 1;
% 				}

% 				message UserResponse {
% 				  string user_id = 1;
% 				  string name = 2;
% 				  int32 age = 3;
% 				}
% 			\end{verbatim}
% 		\end{column}
% 	\end{columns}
% \end{frame}

% \section{Best practices}

% \begin{frame}{Self documenting APIs}
% 	\begin{center}
% 		{\Large\faPencil\quad How do we handle documentation?}
% 	\end{center}
% 	\vspace{1em}
% 	\begin{itemize}
% 		\item REST $\rightarrow$ \alert{OpenAPI}: generated API description, interactive documentation, client SDK, API testing and validation
% 		\item GraphQL $\rightarrow$ \alert{in protocol}: GraphQL can expose its internal resource model schema through a standard API endpoint
% 		\item gRPC $\rightarrow$ \alert{in protocol}: a \texttt{.proto} file is shared across environments and provides both API types and documentation. Messages can be parsed without knowing the protocol definition.
% 	\end{itemize}
% \end{frame}

% \begin{frame}{Input/Model checking}
% 	\begin{center}
% 		{\Large\faPencil\quad What if the focus is performance?}
% 	\end{center}
% 	\vspace{1em}
% 	\begin{itemize}
% 		\item REST $\rightarrow$ \alert{Language or JSON schema}: JSON messages can be type checked using strongly-typed parsing or JSON schema definitions
% 		\item GraphQL $\rightarrow$ \alert{in protocol}: GraphQL requires exchanged messages to match the schema types definitions
% 		\item gRPC $\rightarrow$ \alert{in protocol}: Protobuf-encoded messages are strongly typed and checked during parsing
% 	\end{itemize}
% \end{frame}

% \begin{frame}{Versioning and deprecation}
% 	\begin{center}
% 		{\Large\faPencil\quad How do we handle API versioning and deprecation?}
% 	\end{center}
% 	\vspace{1em}
% 	\begin{itemize}
% 		\item REST $\rightarrow$ \alert{no standard solution}: versioning is achieved through URL, query parameter, header or content negotiation matching.
% 		\item GraphQL \& gRPC $\rightarrow$ \alert{in protocol}: GraphQL and Protobuf have built-in support for deprecation and new methods/attributes can be added without breaking changes. Protobuf also allows \texttt{oneof} type definitions, allowing types extensions.
% 	\end{itemize}
% 	\vspace{1em}
% 	A versioning and deprecation plan should be defined early in development.
% \end{frame}

% \begin{frame}{Logging and error tracking}
% 	\begin{center}
% 		{\Large\faPencil\quad How should we handle logs and track errors?}
% 	\end{center}
% 	\vspace{1em}
% 	Logging and tracing are very language dependent, but some common solutions are available:
% 	\begin{itemize}[<+->]
% 		\item \alert{Strict adherence} to \alert{HTTP status codes} meaning and \alert{JSON structured errors} provide additional context on the issues
% 		\item Logs can be used to correlate errors and to better understand the system and its users. Log information on \alert{request evaluation}, including all sub-evaluations i.e., time, order, \ldots
% 		\item \alert{Span-based logging} can help tagging logs with contextual information
% 		\item \alert{Logs and errors collection services} (Sentry, Elasticsearch, Graylog, \ldots) provide you with a centralized system to store, query and analyze the data obtained by your services and can help automate error managment
% 		\item \alert{Metrics logging} solutions such as Prometheus and OpenTelemetry can help you collect and analyze performance data over time
% 	\end{itemize}
% \end{frame}

% \begin{frame}{Services rollout}
% 	\begin{center}
% 		{\Large\faPencil\quad How to handle changes in services implementation?}
% 	\end{center}
% 	\vspace{1em}
% 	\begin{itemize}[<+->]
% 		\item \alert{Partial rollout} of a new version using \alert{sticky sessions} allows you to handle A-B testing on a large scale
% 		\item \alert{Automated regression testing} prevents most common development errors. \alert{Performance-focused regression} testing gives you a clear idea on the impact of the recent changes
% 	\end{itemize}
% \end{frame}

% \backmatter

% \section{Research in SESAR Lab}

% \begin{frame}{Research topics}

% 	Working on:
% 	\begin{itemize}
% 		\item Formal methods for Security/QoS assurance of infrastructures and services
% 		\item Cloud-edge computing infrastructures (Services, Big Data, Networks, \ldots)
% 		\item 5G services and 5G-enabled edge computing
% 		\item Assurance of the binary pipeline
% 	\end{itemize}

% \end{frame}

% \begin{frame}{Project proposals}
% 	\begin{itemize}
% 		\item \textbf{A non-functional property verification framework based on OpenTelemetry}: \quad Developing a verification framework for web services and interfacing it with OpenTelemetry's API.
% 		\item \textbf{Graph-based API composition for web services: Analysis and POC}: \quad Exploring a plugin based solution for OpenAPI and GraphQL based APIs composition.
% 		\item \textbf{Nix-based big data engine containerization}: \quad Maximising the replicability of our big data infrastructure by migrating to a Nix based container solution, using a completely stateless toolchain that reduces dependency risks.
% 	\end{itemize}
% \end{frame}

% \allreferences[Publications]{biblio.bib}

% \begin{frame}{Contacts}{~}
% 	\begin{columns}
% 		\begin{column}{0.6\textwidth}
% 			\centering
% 			\href{mailto:filippo.berto@unimi.it}{\faEnvelope\enskip\texttt{filippo.berto@unimi.it}} \\\vspace{2em}
% 			\href{https://orcid.org/0000-0002-2720-608X}{\aiOrcid}\quad
% 			\href{https://scholar.google.com/citations?user=8IkjOZEAAAAJ}{\aiGoogleScholarSquare}\quad
% 			\href{https://www.researchgate.net/profile/Filippo-Berto-2}{\aiResearchGateSquare}\quad
% 			\href{https://twitter.com/bertof_}{\faTwitter}\quad
% 			\href{https://www.linkedin.com/in/bertof}{\faLinkedin}\quad
% 			\href{https://www.Gitlab.com/in/bertof}{\faGitlab}\quad
% 			\href{https://www.Github.com/in/bertof}{\faGithub}\quad
% 		\end{column}

% 		\begin{column}{0.4\textwidth}
% 			\centering
% 			\qrcode{https://homes.di.unimi.it/berto} \\\vspace{2em}
% 			\url{homes.di.unimi.it/berto}
% 		\end{column}
% 	\end{columns}
% \end{frame}

\end{document}




